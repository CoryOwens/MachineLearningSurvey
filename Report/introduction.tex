%----------------------------------------------------------------------------------------
%	INTRODUCTION
%----------------------------------------------------------------------------------------

\section*{Introduction} % Major section
\addcontentsline{toc}{section}{Introduction}

According to Dr. Parag Kulkarni, ``Machine learning is the study of methods for programming computers to learn.'' \cite{kulkarni} With traditional algorithms, the program is given a repeatable process to apply for any and all problems to which it is applied. The solution has to be explicitly outlined for the program to work. With machine learning, the program is, instead, given a method to learn from data, and use that to derive its own method for solving associated problems. 

Machine learning is therefore most useful in solving problems which may not be easily generalized, or which may have some degree of variation. Kulkarni highlights four such classes of problems: 

\begin{enumerate}  
\item Repetitive tasks with small variation but which require very high levels of precision. For example, automated manufacturing.
\item Problems which, for humans, the knowledge is tacit. For example, speech recognition and language understanding.
\item Rapidly changing problems. For example, recognition and filtering of spam email. 
\item Applications which must be customized for each individual user. For example, a personal assistant program. 
\end{enumerate}

Machine learning algorithms can be categorized in two primary ways. The first is to categorize them by the types of data they deal with. There are four common categories: 

\begin{enumerate}  
\item Supervised Learning, where the machine is trained on labeled example data.
\item Unsupervised Learning, where the machine is given unlabeled data and asked to find underlying order and patterns.
\item Semisupervised Learning, where the machine is given a mix of labels and unlabeled data.
\item Reinforcement Learning, where the machine is tasked with producing a solution, that solution is then rated or graded and then incorporated back into the machine as data. 
\end{enumerate}

In section 1, this paper will discuss these data categories in more detail. 

The second way to categorized machine learning algorithms is by the output or goal of the algorithm. Here, there are three common categories:

\begin{enumerate}  
\item Classification or Decision machines, where the output is discrete.
\item Regression, where the output is continuous.
\item Clustering, where the output is some set of groupings of data.
\end{enumerate}

In section 2, this paper will discuss these output categories in more detail.

There is a large number of machine learning algorithms. In section 3, this paper will discuss five of the most popular machine learning algorithms

\begin{enumerate}  
\item Linear/Polynomial Regression
\item Logistic Regression
\item Decision Trees
\item Naive Bayes Classifiers
\item Neural Networks
\end{enumerate}

\newpage

%------------------------------------------------
